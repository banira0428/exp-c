\documentclass[11pt]{jsarticle}


% 画像
\usepackage[dvipdfmx]{graphicx}


\begin{document}

\title{情報工学実験C ネットワークプログラミング}
\author{
氏名: 山田 敬汰 (Yamada,Keita) \\
学生番号: 09430559
}
\date{提出日: \today \\   
      締切日: 2021年1月7日 \\}  
\maketitle

\section{クライアント・サーバモデルでのデータ通信}

今回の実験では,分散システムの基本的な形式であるクライアントサーバモデルを理解するために,
TCP/IP , UDP/IP で通信を行うクライアントサーバモデルのプログラムを作成した.

ここでは,プログラムの基礎の部分にあたる,クライアントとサーバ間での通信が
どのような手順で行われているかについて詳しく解説する.

まず初めに,クライアントサーバモデルにおける通信の概要について説明する.クライアントサーバモデルでは,クライアント側の計算機のプロセスが,
サーバ側のプロセスに対してメッセージを送信し,サーバ側が受け取ったメッセージを解釈し,適切な応答を返すことで通信を成立させている.
ここでいう「メッセージ」とはクライアントとサーバの間で予め決めておいた規約(プロトコル)に沿って構築されたテキストである.
(例: http プロトコルでは [メソッド名] [エンドポイント] [改行コード]の順に入力する)

次に,クライアントとサーバが通信する際の具体的な手順について時系列順に説明する.

まず,クライアント側での通信手順は以下の通りである.(通信相手のドメイン名は既知とする)

\begin{enumerate}
      \item DNSサーバに問い合わせを送り,通信相手のドメイン名に対応するIPアドレスの値を取得する.
      \item 通信相手(サーバ)と情報をやり取りするためのソケット(ファイルディスクリプタ)を作成する.
      \item 通信相手(サーバ)との接続を確立する.(クライアント側のソケットとサーバ側のソケットとの対応づけを行う)
      \item プロトコルに沿ったメッセージを構築し,サーバにメッセージを送信する.
      \item サーバからの応答を待機する.
      \item サーバから送られてきたメッセージを受信する.
      \item 通信に使用したソケットを削除する.
\end{enumerate}

そして,サーバ側での通信手順は以下の通りである.

\begin{enumerate}
      \item 通信相手(クライアント)と情報をやり取りするためのソケットを作成する.
      \item ソケットにメタデータを付与する.(どのポートで待ち受けるか,どのIPアドレスと接続するのか等)
      \item ソケットの監視をOSに要求する.(作成したソケットに対する接続要求が行われることをOSに対して通知する)
      \item クライアントからの接続要求を受け入れる.
      \item クライアントから送られてきたメッセージを受信する.
      \item 受信したメッセージに対して何らかの処理を行う.
      \item プロトコルに沿ったメッセージを構築し,クライアントにメッセージを送信する.
\end{enumerate}

\section{プログラムの作成方針}

ここでは,今回作成したプログラムの作成方針について,クライアントサイドとサーバサイドに分けて解説する.

\subsection{名簿管理プログラム(クライアントサイド)の作成方針}

クライアント側のプログラムをおおよそ以下の部分から作成するようにした.以下に,それぞれについての作成方針について述べる.

\begin{enumerate}
      \item サーバとの通信部
      \item 標準入力の解析部
      \item コマンド処理部
\end{enumerate}

\subsubsection{サーバとの通信部}
サーバとの通信部では,前章で説明したクライアント側での通信手順を,システムコールを呼び出すことによって実現する.
なお,今回のプログラムでは,通信部分の一連の流れをリクエスト文字列とレスポンス文字列(を格納するポインタ)を引数とする関数として実装し,
プログラム内から参照しやすいようにする.

\subsubsection{標準入力の解析部}
標準入力の解析部では,キーボードやファイルから入力された一行分の入力を解析し,次にどの処理を行うべきかの条件分岐を行う.
当初は,クライアント側では解析を行わず入力をそのままサーバーに向けて送信する,という実装で実現しようとしていたが,クライアントの終了コマンド({\tt \%Q})
等,サーバ側に送らずともコマンドに該当する処理が完了する場合,無意味な通信をすることになるため,クライアント側でも入力の解析を行うようにした.

\subsubsection{コマンド処理部}
コマンド処理部では,解析した入力の先頭文字が{\tt \%}
だった場合に,その直後の文字に応じた処理を行っている.今回のプログラムでは{\tt \%Q, \%C, \%P, \%R, \%W, \%H}
の6つのコマンドを実装した.これらのコマンドの詳しい動作については後述する.

\subsection{名簿管理プログラム(サーバサイド)の作成方針}

サーバ側のプログラムをおおよそ以下の部分から作成するようにした.以下に,それぞれについての作成方針について述べる.

\begin{enumerate}
      \item クライアントとの通信部
      \item リクエスト解析部
      \item コマンド処理部
\end{enumerate}

\subsubsection{クライアントとの通信部}
クライアントとの通信部では,前節で説明したサーバ側での通信手順を,システムコールを呼び出すことによって実現する.
今回の名簿管理プログラムの場合は,一行ずつ入力を待つループの部分が,クライアント側からのメッセージ受信を待つループに置き換わっていると言える.

\subsubsection{リクエスト解析部}
リクエスト解析部では,クライアントから送られてきたメッセージを解析し,条件分岐を行う.クライアント側でも解析を行っているにも関わらず,
サーバ側で再度解析を行う必要があるのは,今回の名簿管理プログラムにおける通信プロトコルでは,クライアント・サーバ間のやり取りが文字列で行われるからである.
(メッセージを送信する際には解析情報が失われている)

\subsubsection{コマンド実現部}
コマンド実現部では,クライアント側と同様,解析した入力の先頭文字が{\tt \%}だった場合に,その直後の文字に応じた処理を行っている.
なお,サーバ側では{\tt \%C, \%P, \%W, \%H}の4種類のコマンドを実行することが可能である.これらのコマンドの詳細については後述する.

\section{プログラムの説明}

ここでは,前章で説明したプログラムの作
成方針における分類に基づいて,それらの具体的な実装について説明する.

\subsection{名簿管理プログラム(クライアントサイド)の説明}

\subsubsection{サーバとの通信部}

サーバとの通信を以下に示す処理を実装することによって実現する.また,該当部分のソースコードは \ref{sec:client.c}節に添付する.

\begin{enumerate}
      \item {\tt gethostbyname}関数を呼び出し,通信先のIPアドレスを取得する.今回の課題では,自分のPC内でクライアントプログラムとサーバプログラムを同時に起動して実験を行うという前提のもと,ドメイン名は{\tt localhost}で固定している.(ループバックアドレスである {\tt 127.0.0.1}が得られる.)
      \item {\tt socket}関数を呼び出し,通信用のファイルディスクリプタ番号を取得する.
      \item {\tt connect}関数を呼び出し,サーバとの接続を確立する.なお,ここで引数として渡す{\tt sa}構造体には,ポート番号やIPアドレスなどの接続先に関する情報を格納している.
      \item {\tt send}関数を呼び出し,サーバへメッセージを送信する.ここでは,関数の呼び出し側がメッセージ長を気にしなくてもいいようにするために,送信したバイト数を確認し,それが{\tt BUF\_SIZE}と一致する限り送信を繰り返すようにしている.({\tt BUF\_SIZE}で分割して送信するようにしている.)
      \item {\tt receive}関数を呼び出し,サーバから送られてきたメッセージを受信する.こちらも,受信するメッセージ長が{\tt BUF\_SIZE}を超えても良いように,受信したバイト数が{\tt BUF\_SIZE}と一致する限り受信を繰り返すようにしている.
\end{enumerate}

\subsubsection{標準入力の解析部}

標準入力の解析部では,これまでの名簿管理プログラム同様,標準入力から一行ずつ入力を読み取り,名簿データが入力された場合はそれをサーバにそのまま送信し,先頭文字が{\tt \%}だった場合は次の文字を見て該当するコマンド処理への条件分岐を行っている.なお,この部分のプログラムは\ref{sec:meibo_client.c}に記載する.

\subsubsection{コマンド処理部}

コマンド処理部では,先ほどの解析部での条件分岐に応じて,様々な処理を実行する.今回の名簿管理プログラムでは以下に示すコマンド処理を実装した.


\subsection{名簿管理プログラム(サーバサイド)の説明}

\section{プログラムの使用法}


\section{プログラムの作成過程における考察}
\section{得られた結果に関する考察}
\section{作成したプログラム}

\subsection{名簿管理プログラムの通信部(クライアント)} \label{sec:client.c}

\begin{verbatim}
      #include "client.h"

      int request(char *request, char *response) {
        struct hostent *hp;
        hp = gethostbyname("localhost");
      
        if (hp == NULL) {
          printf("host not found\n");
          return -1;
        }
      
        // create socket
        int soc = socket(AF_INET, SOCK_STREAM, 0);
        if (soc == -1) {
          printf("failed to create socket\n");
          return -1;
        }
      
        struct sockaddr_in sa;
        sa.sin_family = hp->h_addrtype;
        sa.sin_port = htons(PORT_NO);
        bzero((char *)&sa.sin_addr, sizeof(sa.sin_addr));
        memcpy((char *)&sa.sin_addr, (char *)hp->h_addr, hp->h_length);
      
        int connect_result = connect(soc, 
                  (struct sockaddr *)&sa, sizeof(sa));
        if (connect_result == -1) {
          printf("failed to connect\n");
          return -1;
        }
      
        // send request
        int send_result = 0;
        int sended_bytes = 0;
        do {
          send_result = send(soc, request + sended_bytes,
                             fmin(BUF_SIZE, 
                             strlen(request + sended_bytes) + 1), 0);
          sended_bytes += send_result;
          if (send_result == -1) {
            perror("failed to send");
            break;
          }
        } while (send_result == BUF_SIZE);
      
        // receive response
        int recv_result = 0;
        sprintf(response, "");
        do {
          char tmp[BUF_SIZE] = "";
          recv_result = recv(soc, tmp, BUF_SIZE, 0);
          strcat(response, tmp);
          if (recv_result == -1) {
            printf("failed to receive\n");
            return -1;
          }
        } while (recv_result == BUF_SIZE);
      
        close(soc);
      
        return 0;
      }
\end{verbatim}

\subsection{名簿管理プログラムの標準入力解析部(クライアント)} \label{sec:meibo_client.c}

\begin{verbatim}
#include "meibo_client.h"

int main() {
  char line[INPUT_MAX + 1];
  while (get_line_fp(stdin, line)) {
    parse_line(line);
  }
  return 1;
}

void parse_line(char *line) {
  if (*line == '%') {
    char *exec[] = {"", "", "", "", ""};
    split(line + 1, exec, ' ', 5);
    exec_command_str(exec);
  } else {
    char response[BUF_SIZE];
    request(line, response);
  }
}

void exec_command_str(char *exec[]) {
  if (!strcmp("Q", exec[0])) {
    cmd_quit();
  } else if (!strcmp("C", exec[0])) {
    cmd_check();
  } else if (!strcmp("P", exec[0])) {
    cmd_print(exec[1]);
  } else if (!strcmp("R", exec[0])) {
    cmd_read(exec[1]);
  } else if (!strcmp("W", exec[0])) {
    cmd_write(exec[1]);
  } else if (!strcmp("H", exec[0])) {
    cmd_history();
  } else {
    fprintf(stderr, "Invalid command %s: ignored.\n", exec[0]);
  }
  return;
}

\end{verbatim}

\subsection{名簿管理プログラムのコマンド処理部(クライアント)} \label{sec:commands.c}

\begin{verbatim}
      #include "commands.h"

void cmd_quit() { exit(0); }

void cmd_check() {
  char response[BUF_SIZE];
  request("%C", response);
  printf("%s profile(s)\n", response);
}

void cmd_read(char *path) {
  FILE *fp;
  fp = fopen(path, "r");

  if (fp == NULL) {
    fprintf(stderr, "Could not open file: %s\n", path);
    return;
  }

  char line[INPUT_MAX + 1];
  while (get_line_fp(fp, line)) { /* fp を引数に追加 */
    char response[BUF_SIZE];
    request(line, response);
  }

  fclose(fp);

  printf("success: cmd_read\n");

  return;
};

void cmd_write(char *path) {
  FILE *fp;
  if (*path == '\0') {
    path = "data/output.csv";
  }

  fp = fopen(path, "w");
  if (fp == NULL) {
    fprintf(stderr, "Could not open file: %s\n", path);
    return;
  }
    
  char response[RESPONSE_BUF_SIZE];
  request("%W", response);

  fprintf(fp, response);
  fclose(fp);

  printf("write success: %s\n", path);

  return;
}

void cmd_print(char *num) {
  char query[BUF_SIZE];
  sprintf(query,"%%P %s", num);
  char response[RESPONSE_BUF_SIZE];
  request(query, response);
  printf(response);
  return;
}

void cmd_history() {
  char query[BUF_SIZE] = "%H";
  char response[BUF_SIZE];
  
  request("%H", response);
  printf("%s\n", response);
}

\end{verbatim}



\end{document}