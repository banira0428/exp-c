\documentclass[11pt]{jsarticle}


% 画像
\usepackage[dvipdfmx]{graphicx}


\begin{document}

\title{情報工学実験C ネットワークプログラミング}
\author{
氏名: 山田 敬汰 (Yamada,Keita) \\
学生番号: 09430559
}
\date{提出日: \today \\   
      締切日: 2021年1月7日 \\}  
\maketitle

\section{クライアント・サーバモデルでのデータ通信}

今回の実験では,分散システムの基本的な形式であるクライアントサーバモデルを理解するために,
TCP/IP , UDP/IP で通信を行うクライアントサーバモデルのプログラムを作成した.

ここでは,プログラムの基礎の部分にあたる,クライアントとサーバ間での通信が
どのような手順で行われているかについて詳しく解説する.

まず初めに,クライアントサーバモデルにおける通信の概要について説明する.クライアントサーバモデルでは,クライアント側の計算機のプロセスが,
サーバ側のプロセスに対してメッセージを送信し,サーバ側が受け取ったメッセージを解釈し,適切な応答を返すことで通信を成立させている.
ここでいう「メッセージ」とはクライアントとサーバの間で予め決めておいた規約(プロトコル)に沿って構築されたテキストである.
(例: http プロトコルでは [メソッド名] [エンドポイント] [改行コード]の順に入力する)

次に,クライアントとサーバが通信する際の具体的な手順について時系列順に説明する.

クライアント側での通信手順は以下の通りである.(通信相手のドメイン名は既知とする)

\begin{enumerate}
      \item DNSサーバに問い合わせを送り,通信相手のドメイン名に対応するIPアドレスの値を取得する.
      \item 通信相手(サーバ)と情報をやり取りするためのソケット(ファイルディスクリプタ)を作成する.
      \item 通信相手(サーバ)との接続を確立する.(クライアント側のソケットとサーバ側のソケットとの対応づけを行う)
      \item プロトコルに沿ったメッセージを構築し,サーバにメッセージを送信する.
      \item サーバからの応答を待機する.
      \item サーバから送られてきたメッセージを受信する.
      \item 通信に使用したソケットを削除する.
\end{enumerate}

サーバ側での通信手順は以下の通りである.

\begin{enumerate}
      \item 通信相手(クライアント)と情報をやり取りするためのソケットを作成する.
      \item ソケットにメタデータを付与する.(どのポートで待ち受けるか,どのIPアドレスと接続するのか等)
      \item ソケットの監視をOSに要求する.(作成したソケットに対する接続要求が行われることをOSに対して通知する)
      \item クライアントからの接続要求を受け入れる.
      \item クライアントから送られてきたメッセージを受信する.
      \item 受信したメッセージに対して何らかの処理を行う.
      \item プロトコルに沿ったメッセージを構築し,クライアントにメッセージを送信する.
\end{enumerate}

\section{名簿管理プログラムの作成方針}

プログラムをおおよそ以下の部分から作成するようにした.以下に,それぞれについての作成方針について述べる.

\begin{enumerate}
      \item 
\end{enumerate}



\section{名簿管理プログラムの説明}



\section{名簿管理プログラムの使用法}


\section{名簿管理プログラムの作成過程における考察}
\section{得られた結果に関する考察}
\section{作成したプログラム}

\end{document}