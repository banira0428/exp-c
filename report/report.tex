\documentclass[11pt]{jsarticle}


% 画像
\usepackage[dvipdfmx]{graphicx}


\begin{document}

\title{情報工学実験C ネットワークプログラミング}
\author{
氏名: 山田 敬汰 (Yamada,Keita) \\
学生番号: 09430559
}
\date{提出日: \today \\   
      締切日: 2021年1月7日 \\}  
\maketitle

\section{クライアント・サーバモデルでのデータ通信}

今回の実験では,分散システムの基本的な形式であるクライアントサーバモデルを理解するために,
TCP/IP , UDP/IP で通信を行うクライアントサーバモデルのプログラムを作成した.

ここでは,プログラムの基礎の部分にあたる,クライアントとサーバ間での通信が
どのような手順で行われているかについて詳しく解説する.

まず初めに,クライアントサーバモデルにおける通信の概要について説明する.クライアントサーバモデルでは,クライアント側の計算機のプロセスが,
サーバ側のプロセスに対してメッセージを送信し,サーバ側が受け取ったメッセージを解釈し,適切な応答を返すことで通信を成立させている.
ここでいう「メッセージ」とはクライアントとサーバの間で予め決めておいた規約(プロトコル)に沿って構築されたテキストである.
(例: http プロトコルでは [メソッド名] [エンドポイント] [改行コード]の順に入力する)

次に,クライアントとサーバが通信する際の具体的な手順について時系列順に説明する.

まず,クライアント側での通信手順は以下の通りである.(通信相手のドメイン名は既知とする)

\begin{enumerate}
      \item DNSサーバに問い合わせを送り,通信相手のドメイン名に対応するIPアドレスの値を取得する.
      \item 通信相手(サーバ)と情報をやり取りするためのソケット(ファイルディスクリプタ)を作成する.
      \item 通信相手(サーバ)との接続を確立する.(クライアント側のソケットとサーバ側のソケットとの対応づけを行う)
      \item プロトコルに沿ったメッセージを構築し,サーバにメッセージを送信する.
      \item サーバからの応答を待機する.
      \item サーバから送られてきたメッセージを受信する.
      \item 通信に使用したソケットを削除する.
\end{enumerate}

そして,サーバ側での通信手順は以下の通りである.

\begin{enumerate}
      \item 通信相手(クライアント)と情報をやり取りするためのソケットを作成する.
      \item ソケットにメタデータを付与する.(どのポートで待ち受けるか,どのIPアドレスと接続するのか等)
      \item ソケットの監視をOSに要求する.(作成したソケットに対する接続要求が行われることをOSに対して通知する)
      \item クライアントからの接続要求を受け入れる.
      \item クライアントから送られてきたメッセージを受信する.
      \item 受信したメッセージに対して何らかの処理を行う.
      \item プロトコルに沿ったメッセージを構築し,クライアントにメッセージを送信する.
\end{enumerate}

\section{プログラムの作成方針}

ここでは,今回作成したプログラムの作成方針について,クライアントサイドとサーバサイドに分けて解説する.

\subsection{名簿管理プログラム(クライアントサイド)の作成方針}

クライアント側のプログラムをおおよそ以下の部分から作成するようにした.以下に,それぞれについての作成方針について述べる.

\begin{enumerate}
      \item サーバとの通信部
      \item 標準入力の解析部
      \item コマンド処理部
\end{enumerate}

\subsubsection{サーバとの通信部}
サーバとの通信部では,前節で説明したクライアント側での通信手順を,システムコールを呼び出すことによって実現する.
なお,今回のプログラムでは,通信部分の一連の流れをリクエスト文字列とレスポンス文字列(を格納するポインタ)を引数とする関数として実装し,
プログラム内から参照しやすいようにする.

\subsubsection{標準入力の解析部}
標準入力の解析部では,キーボードやファイルから入力された一行分の入力を解析し,次にどの処理を行うべきかの条件分岐を行う.
当初は,クライアント側では解析を行わず入力をそのままサーバーに向けて送信する,という実装で実現しようとしていたが,クライアントの終了コマンド({\tt \%Q})
等,サーバ側に送らずともコマンドに該当する処理が完了する場合,無意味な通信をすることになるため,クライアント側でも入力の解析を行うようにした.

\subsubsection{コマンド処理部}
コマンド処理部では,解析した入力の先頭文字が{\tt \%}
だった場合に,その直後の文字に応じた処理を行っている.今回のプログラムでは{\tt \%Q, \%C, \%P, \%R, \%W, \%H}
の6つのコマンドを実装した.これらのコマンドの詳しい動作については後述する.

\subsection{名簿管理プログラム(サーバサイド)の作成方針}

サーバ側のプログラムをおおよそ以下の部分から作成するようにした.以下に,それぞれについての作成方針について述べる.

\begin{enumerate}
      \item クライアントとの通信部
      \item リクエスト解析部
      \item コマンド処理部
\end{enumerate}

\subsubsection{クライアントとの通信部}
クライアントとの通信部では,前節で説明したサーバ側での通信手順を,システムコールを呼び出すことによって実現する.
今回の名簿管理プログラムの場合は,一行ずつ入力を待つループの部分が,クライアント側からのメッセージ受信を待つループに置き換わっていると言える.

\subsubsection{リクエスト解析部}
リクエスト解析部では,クライアントから送られてきたメッセージを解析し,条件分岐を行う.クライアント側でも解析を行っているにも関わらず,
サーバ側で再度解析を行う必要があるのは,今回の名簿管理プログラムにおける通信プロトコルでは,クライアント・サーバ間のやり取りが文字列で行われるからである.
(メッセージを送信する際には解析情報が失われている)

\subsubsection{コマンド実現部}
コマンド実現部では,クライアント側と同様,解析した入力の先頭文字が{\tt \%}だった場合に,その直後の文字に応じた処理を行っている.
なお,サーバ側では{\tt \%C, \%P, \%W, \%H}の4種類のコマンドを実行することが可能である.これらのコマンドの詳細については後述する.

\section{プログラムの説明}



\section{プログラムの使用法}


\section{プログラムの作成過程における考察}
\section{得られた結果に関する考察}
\section{作成したプログラム}

\end{document}