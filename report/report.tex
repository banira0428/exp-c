\documentclass[11pt]{jsarticle}


% 画像
\usepackage[dvipdfmx]{graphicx}


\begin{document}

\title{情報工学実験C ネットワークプログラミング}
\author{
氏名: 山田 敬汰 (Yamada,Keita) \\
学生番号: 09430559
}
\date{提出日: \today \\   
      締切日: 2021年1月12日 \\}  
\maketitle

\section{クライアント・サーバモデルでのデータ通信}

今回の実験では,分散システムの基本的な形式であるクライアントサーバモデルを理解するために,
TCP/IP , UDP/IP で通信を行うクライアントサーバモデルのプログラムを作成した.

ここでは,プログラムの基礎の部分にあたる,クライアントとサーバ間での通信が
どのような手順で行われているかについて詳しく解説する.

まず初めに,クライアントサーバモデルにおける通信の概要について説明する.クライアントサーバモデルでは,クライアント側の計算機のプロセスが,
サーバ側のプロセスに対してメッセージを送信し,サーバ側が受け取ったメッセージを解釈し,適切な応答を返すことで通信を成立させている.
ここでいう「メッセージ」とはクライアントとサーバの間で予め決めておいた規約(プロトコル)に沿って構築されたテキストである.
(例: http プロトコルでは [メソッド名] [エンドポイント] [改行コード]の順に入力する)

次に,クライアントとサーバが通信する際の具体的な手順について時系列順に説明する.

まず,クライアント側での通信手順は以下の通りである.(通信相手のドメイン名は既知とする)

\begin{enumerate}
      \item DNSサーバに問い合わせを送り,通信相手のドメイン名に対応するIPアドレスの値を取得する.
      \item 通信相手(サーバ)と情報をやり取りするためのソケット(ファイルディスクリプタ)を作成する.
      \item 通信相手(サーバ)との接続を確立する.(クライアント側のソケットとサーバ側のソケットとの対応づけを行う)
      \item プロトコルに沿ったメッセージを構築し,サーバにメッセージを送信する.
      \item サーバからの応答を待機する.
      \item サーバから送られてきたメッセージを受信する.
      \item 通信に使用したソケットを削除する.
\end{enumerate}

そして,サーバ側での通信手順は以下の通りである.

\begin{enumerate}
      \item 通信相手(クライアント)と情報をやり取りするためのソケットを作成する.
      \item ソケットにメタデータを付与する.(どのポートで待ち受けるか,どのIPアドレスと接続するのか等)
      \item ソケットの監視をOSに要求する.(作成したソケットに対する接続要求が行われることをOSに対して通知する)
      \item クライアントからの接続要求を受け入れる.
      \item クライアントから送られてきたメッセージを受信する.
      \item 受信したメッセージに対して何らかの処理を行う.
      \item プロトコルに沿ったメッセージを構築し,クライアントにメッセージを送信する.
\end{enumerate}

\section{プログラムの作成方針}

ここでは,今回作成したプログラムそれぞれ(名簿管理プログラム,発展課題として作成した簡易認証プログラム)の作成方針について,クライアントサイドとサーバサイドに分けて解説する.

\subsection{名簿管理プログラム(クライアントサイド)の作成方針}

クライアント側のプログラムをおおよそ以下の部分から作成するようにした.以下に,それぞれについての作成方針について述べる.

\begin{enumerate}
      \item サーバとの通信部
      \item 標準入力の解析部
      \item コマンド処理部
\end{enumerate}

\subsubsection{サーバとの通信部}
サーバとの通信部では,前章で説明したクライアント側での通信手順を,システムコールを呼び出すことによって実現する.
なお,今回のプログラムでは,通信部分の一連の流れをリクエスト文字列とレスポンス文字列(を格納するポインタ)を引数とする関数として実装し,
プログラム内から参照しやすいようにする.

\subsubsection{標準入力の解析部}
標準入力の解析部では,キーボードやファイルから入力された一行分の入力を解析し,次にどの処理を行うべきかの条件分岐を行う.
当初は,クライアント側では解析を行わず入力をそのままサーバーに向けて送信する,という実装で実現しようとしていたが,クライアントの終了コマンド({\tt \%Q})
等,サーバ側に送らずともコマンドに該当する処理が完了する場合,無意味な通信をすることになるため,クライアント側でも入力の解析を行うようにした.

\subsubsection{コマンド処理部}
コマンド処理部では,解析した入力の先頭文字が{\tt \%}
だった場合に,その直後の文字に応じた処理を行っている.今回のプログラムでは{\tt \%Q, \%C, \%P, \%R, \%W, \%H}
の6つのコマンドを実装した.これらのコマンドの詳しい動作については後述する.

\subsection{名簿管理プログラム(サーバサイド)の作成方針}

サーバ側のプログラムをおおよそ以下の部分から作成するようにした.以下に,それぞれについての作成方針について述べる.

\begin{enumerate}
      \item クライアントとの通信部
      \item リクエスト解析部
      \item コマンド処理部
\end{enumerate}

\subsubsection{クライアントとの通信部}
クライアントとの通信部では,前節で説明したサーバ側での通信手順を,システムコールを呼び出すことによって実現する.
今回の名簿管理プログラムの場合は,一行ずつ入力を待つループの部分が,クライアント側からのメッセージ受信を待つループに置き換わっていると言える.

\subsubsection{リクエスト解析部}
リクエスト解析部では,クライアントから送られてきたメッセージを解析し,条件分岐を行う.クライアント側でも解析を行っているにも関わらず,
サーバ側で再度解析を行う必要があるのは,今回の名簿管理プログラムにおける通信プロトコルでは,クライアント・サーバ間のやり取りが文字列で行われるからである.
(メッセージを送信する際には解析情報が失われている)

\subsubsection{コマンド実現部}
コマンド実現部では,クライアント側と同様,解析した入力の先頭文字が{\tt \%}だった場合に,その直後の文字に応じた処理を行っている.
なお,サーバ側では{\tt \%C, \%P, \%W, \%H}の4種類のコマンドを実行することが可能である.これらのコマンドの詳細については後述する

\subsection{簡易認証プログラム(クライアント)の作成方針}

今回の実験では,発展課題として簡易的な認証システムを作成した.
具体的には,ユーザ登録,ログイン,ログアウトなどの基本的な操作が実行できるようになっている.
また,今回のプログラムでは,トークンを利用した認証を行なっている.

なお,ここでいう「ユーザ」とは,名前,パスワード,トークンをフィールドとして持つ構造体のことである.

コマンド実現部以外の作成方針に関しては名簿管理プログラムとほぼ同様なので割愛する.

\subsubsection{コマンド実現部}
コマンド実現部では,入力を空白区切りしたものの最初の文字列で条件分岐を行い,各コマンドを実行する.
今回のプログラムでは{\tt Register, Status, Edit, Login, Logout}の5つのコマンドを実装した.これらのコマンドの詳しい動作については後述する.

\subsection{簡易認証プログラム(サーバ)の作成方針}

サーバ側の主な役目は,トークン発行及び会員の管理である.

コマンド実現部以外の作成方針に関しては名簿管理プログラムとほぼ同様なので割愛する.

\subsubsection{コマンド実現部}
コマンド実現部では,入力を空白区切りしたものの最初の文字列で条件分岐を行い,各コマンドを実行する.
今回のプログラムでは{\tt Register, Edit, Login}の3つのコマンドを実装した.これらのコマンドの詳しい動作については後述する.

\section{プログラムの説明}

ここでは,前章で説明したプログラムの作成方針における分類に基づいて,それらの具体的な実装について説明する.

\subsection{名簿管理プログラム(クライアントサイド)の説明}

\subsubsection{サーバとの通信部}

サーバとの通信を以下に示す処理を実装することによって実現する.また,該当部分のソースコードは \ref{sec:client.c}節に添付する.

\begin{enumerate}
      \item {\tt gethostbyname}関数を呼び出し,通信先のIPアドレスを取得する.今回の課題では,自分のPC内でクライアントプログラムとサーバプログラムを同時に起動して実験を行うという前提のもと,ドメイン名は{\tt localhost}で固定している.(ループバックアドレスである {\tt 127.0.0.1}が得られる.)
      \item {\tt socket}関数を呼び出し,通信用のファイルディスクリプタ番号を取得する.
      \item {\tt connect}関数を呼び出し,サーバとの接続を確立する.なお,ここで引数として渡す{\tt sa}構造体には,ポート番号やIPアドレスなどの接続先に関する情報を格納している.
      \item {\tt send}関数を呼び出し,サーバへメッセージを送信する.ここでは,関数の呼び出し側がメッセージ長を気にしなくてもいいようにするために,送信したバイト数を確認し,それが{\tt BUF\_SIZE}と一致する限り送信を繰り返すようにしている.({\tt BUF\_SIZE}で分割して送信するようにしている.)
      \item {\tt receive}関数を呼び出し,サーバから送られてきたメッセージを受信する.こちらも,受信するメッセージ長が{\tt BUF\_SIZE}を超えても良いように,受信したバイト数が{\tt BUF\_SIZE}と一致する限り受信を繰り返すようにしている.
\end{enumerate}

\subsubsection{標準入力の解析部}

標準入力の解析部では,これまでの名簿管理プログラム同様,標準入力から一行ずつ入力を読み取り,名簿データが入力された場合はそれをサーバにそのまま送信し,先頭文字が{\tt \%}だった場合は次の文字を見て該当するコマンド処理への条件分岐を行っている.なお,この部分のプログラムは\ref{sec:meibo_client.c}に添付する.

\subsubsection{コマンド処理部}

コマンド処理部では,先ほどの解析部での条件分岐に応じて,様々な処理を実行する.今回の名簿管理プログラムでは以下に示すコマンド処理を実装した.なお,該当の部分のソースコードを\ref{sec:commands.c}節に添付する.

\begin{itemize}
      \item Qコマンドが呼び出された場合,{\tt exit}関数を実行しプログラムを終了する.
      \item Cコマンドが呼び出された場合,サーバ側にCコマンドを送信し,サーバ側から名簿の件数を取得し表示する.
      \item Rコマンドが呼び出された場合,クライアントの計算機内に存在するファイルの内容を読み取り,サーバ側に新規登録用の名簿データとして一行ごとに送信している.ここで,クライアント側のファイルを読み込むようにしたのは,サーバクライアントモデルでは,一つのサーバに対してクライアントが複数存在することが一般的であり,各クライアントからデータを収集するというユースケースを想定したからである.
      \item Wコマンドが呼び出された場合,サーバ側にWコマンドを送信し,サーバ側から名簿データを全件取得しクライアント側に存在するファイルにCSVデータとして書き込む.ここで,クライアント側のファイルに対する書き込みを行うようにしているのは,Rコマンドの説明でも述べたとおり,一つのサーバに対してクライアントが複数存在するという状況を想定したからである.(データの永続化の役割をサーバ側が担っている)
      \item Pコマンドが呼び出された場合,サーバ側にPコマンドを送信し,サーバ側から指定した件数のデータを受け取りプロンプトに表示する.なお,サーバからの応答の形式に違いこそあれ,「サーバ側から複数件のデータを取得し処理する」という見方をした場合,Wコマンドと似たような処理をしているとも言える.
      \item Hコマンドが呼び出された場合,サーバ側にHコマンドを送信し,サーバ側からコマンドの入力履歴のうち最新10
      件を取得し表示する.このコマンドを実装した理由は,一つのクライアントだけではサーバ側で行われた操作の全容を把握することができなかったからである.(つまり,このコマンドによってクライアントAが,クライアントBがサーバに対して行った操作を知ることができる)
\end{itemize}

\subsection{名簿管理プログラム(サーバ側)の説明}

\subsubsection{クライアントとの通信部}

クライアントとの通信を以下に示す処理を実装することによって実現する.また,該当部分のソースコードは \ref{sec:server.c}節に添付する.

\begin{enumerate}
      \item {\tt socket}関数を呼び出し,クライアントとの通信用のファイルディスクリプタ(ソケット)を作成する.
      \item {\tt bind}関数を呼び出し,先ほど作成したソケットとサーバ側の特定のポートを紐付ける.
      \item {\tt listen}関数を呼び出し,OSに対してソケットに接続要求がなされることを通知する.
      \item クライアント側からの接続要求があるまで待機し,接続要求(クライアント側の{\tt connect}関数)が来たタイミングで{\tt accept}関数を呼び出し,接続を確立する.
      \item クライアント側からメッセージが送信されるまで待機し,メッセージが送られてきたタイミングで{\tt recv}関数を呼び出しクライアント側からのメッセージを配列に格納する.送られてきたメッセージのサイズがバッファサイズよりも大きい場合に備えて,クライアント側同様,送られてきたデータサイズに応じて{\tt recv}を複数回繰り返すようにしている.
      \item 送られてきたメッセージの解析等を行う.(詳しくは後述)
      \item {\tt send}関数を呼び出し,クライアント側に応答メッセージを返す.こちらも,クライアント側の{\tt send}部分と同じように送信するメッセージのサイズに応じて複数回{\tt send}を繰り返す実装とした.
\end{enumerate}

\subsubsection{リクエスト解析部}

リクエスト解析部では,クライアントから送られてきたメッセージを解析し,新規名簿データの追加や,各コマンド処理に対する条件分岐を行っている.去年作った名簿管理プログラムと大きく異なる点としては,Hコマンドの存在により,コマンドの入力があった場合にそれを入力履歴として配列に保存するようにしている.(最新の数件を保存するキュー構造のリスト) なお,該当の部分のソースコードは\ref{sec:parse_request.c}節に添付する.

\subsubsection{コマンド実現部}

コマンド処理部では,先ほどの解析部での条件分岐に応じて,様々な処理を実行する.今回の名簿管理プログラムでは以下に示すコマンド処理を実装した.また,該当部分のソースコードを\ref{sec:commands_server.c}節に添付する.

\begin{itemize}
      \item Cコマンドが呼び出された場合,名簿の件数をクライアント側への応答として返す.
      \item Wコマンドが呼び出された場合,保存されている名簿データをCSV形式のテキストにフォーマットし,クライアント側への応答として返す.
      \item Pコマンドが呼び出された場合,保存されている名簿データを与えられた引数の数によって絞り込み(上限,順序等),絞り込んだ結果をコンソール表示用のテキストにフォーマットしてクライアント側への応答として返す.
      \item Hコマンドが呼び出された場合,リクエスト解析部で保存しておいた入力履歴の内容をクライアント側への応答として返す.
\end{itemize}

\subsection{簡易認証プログラム(クライアント側)の説明}

ここでは,簡易認証プログラムのクライアント側のコマンド実現部について詳しく解説する.
(通信部分や入力の解析部は名簿管理プログラムとほぼ同様なので割愛する.)
また,該当部分のソースコードは,\ref{sec:auth_commands_client.c}節に添付する.

\begin{itemize}
  \item コマンド名:Register
  \item 用途:新規ユーザ作成
  \item 引数:ユーザ名,パスワード,確認用パスワード
  \item 返り値(登録に成功した場合):サーバから付与されたトークン(ランダムな文字列)
  \item 返り値(登録に失敗した場合):サーバ側で発行されたエラーメッセージ
  \item 実行例: {\tt Register keita password password}
\end{itemize}

Register コマンドでは,必要な情報を引数として入力し,ユーザ登録を行う.
サーバ側で入力された情報が問題ないと判断された場合,返り値としてサーバ側から付与されたトークンをグローバル変数として保持する.(認証済みの状態になる)

\begin{itemize}
  \item コマンド名:Login
  \item 用途:ユーザログイン
  \item 引数:ユーザ名,パスワード
  \item 返り値:(認証に成功した場合)ユーザの持っているトークン
  \item 返り値(認証に失敗した場合):サーバ側で発行されたエラーメッセージ
  \item 例外:{\tt Already\_Authenticated }(認証済みの状態でログインコマンドを入力した場合)
  \item 実行例:{\tt Login keita password}
\end{itemize}

Login コマンドでは,既に会員登録を済ませたユーザ情報がサーバ内に存在するときに,その情報を用いて認証を行う.
サーバ側で問題なく認証ができた場合,会員登録時同様返り値としてサーバ側から付与されたトークンをグローバル変数として保持する.

\begin{itemize}
  \item コマンド名:Edit
  \item 用途:ユーザ名編集(上書き)
  \item 引数:ユーザ名
  \item 返り値:(編集に成功した場合)成功を示すメッセージ
  \item 返り値(編集に失敗した場合):サーバ側で発行されたエラーメッセージ
  \item 実行例:{\tt Edit yamada}
\end{itemize}

Edit コマンドは,クライアントがログイン中の時のみ有効なコマンド(保持しているトークン情報をサーバ側へ送るメッセージに付与する)であり,
サーバ上で保持されているユーザ情報の内,ユーザ名を上書きする.ここでユーザの名前を更新すると,
次回以降のログインの際に入力する内容を変更する必要がある.

\begin{itemize}
  \item コマンド名:Status
  \item 用途:認証状態の確認
  \item 引数:無し
  \item 返り値:{\tt Authenticated}(認証済み)または{\tt Guest}(未認証)
  \item 実行例:{\tt Status}
\end{itemize}

Status コマンドでは,クライアント側のプログラムが現在ログイン中かどうかを確認することが可能である.
具体的には,トークンを保持しているかどうかで認証状態を判断している.このコマンドは主にデバッグ用として作成した.

\begin{itemize}
  \item コマンド名:Logout
  \item 用途:ログアウト
  \item 引数:無し
  \item 返り値:成功を示すメッセージ
  \item 例外:{\tt Not\_Authenticated }(未認証の状態でログアウトコマンドを入力した場合)
  \item 実行例:{\tt Logout}
\end{itemize}

Logout コマンドでは,現在保持しているトークンを破棄し,未認証の状態へと変化させる.

\subsection{簡易認証プログラム(サーバ側)の説明}

ここでは,簡易認証プログラムのサーバ側のコマンド実現部について説明する.

なお,該当部分のソースコードは \ref{sec:auth_commands_server.c}節に添付する.

\begin{itemize}
  \item コマンド名:Register
  \item 用途:新規ユーザ作成(トークン生成)
  \item 引数:ユーザ名,パスワード,確認用パスワード
  \item 返り値(登録に成功した場合):トークン(ランダムな文字列)
  \item 返り値(登録に失敗した場合):エラーメッセージ
  \item 例外1: {\tt Already\_Registered}(既に登録済みのユーザ名で新規登録を行おうとした場合)
  \item 例外2: {\tt Password\_Unmatched}(パスワードと確認用パスワードが異なっていた場合)
\end{itemize}

サーバ側の Register コマンドでは,主にクライアント側で入力された登録情報に対してバリデーションを行う.
入力に問題がない場合はユーザ情報及び認証したことの証明であるトークンを新しく作成し,トークンをクライアント側に返却している.

なお,ここでいう「トークン」とは,ランダムな英数字を64桁並べたものである.({\tt generate\_token}関数にて生成)

\begin{itemize}
  \item コマンド名:Login
  \item 用途:ユーザログイン
  \item 引数:ユーザ名,パスワード
  \item 返り値(認証に成功した場合):ユーザの持つトークン
  \item 返り値(認証に失敗した場合):エラーメッセージ
  \item 例外1: {\tt User\_Not\_Found}(存在しないユーザ名でログインを行おうとした場合)
  \item 例外2: {\tt Password\_Is\_Incorrectedd}(入力されたパスワードが会員情報のパスワードと一致しない場合)
\end{itemize}

サーバ側の Login コマンドでは,クライアント側で入力されたログイン情報を検証し,
それが正しければ,ログインできたことの証明としてユーザの持つトークンをクライアント側に返却する.

今回のプログラムでは,デバッグのしやすさを優先するため,
ユーザ名が異なる時とパスワードが異なる場合とで別々のエラーメッセージを表示するようにしているが,
現実的には攻撃者の手がかりとなり得る(例:{\tt User\_Not\_Found}例外の有無で任意のユーザ名の存在を確かめることができる)ため,
安全性と利便性はトレードオフになることを留意しておくべきである.

\begin{itemize}
  \item コマンド名:Edit
  \item 用途:ユーザ名編集(上書き)
  \item 引数:ユーザ名,トークン
  \item 返り値(編集に成功した場合):成功メッセージ
  \item 返り値(編集に失敗した場合):エラーメッセージ
  \item 例外: {\tt Invalid\_Token}(不正なトークンが入力された場合)
\end{itemize}

サーバ側ではクライアント側から送られてきたトークン情報を検証し,問題がなければそのトークンに該当するユーザ名を更新する.

「トークン」をユーザ特定のために用いているのは,仮に変更前のユーザ名を用いて名前を変更したいユーザを特定するような実装にした場合,
本人以外(パスワードを知らない人)にもユーザ名の変更ができてしまうからである.

\section{プログラムの使用法}

\subsection{名簿管理プログラムの使用法}

今回の実験で作成した名簿管理プログラムは,クライアント側のプログラムに標準入力でCSV形式の名簿データまたは,{\tt \%}から始まるコマンド入力を受け付け,場合によってはサーバ側のプログラムとも通信を行い,処理結果を標準出力に出力する.

具体例として,以下に示す入力を与えた際に得られる出力について説明する.

{\small
\begin{verbatim}
%C
5100046,The Bridge,1845-11-2,14 Seafield Road Longman Inverness,SEN Unit 2.0 Open
%C
%R data/sample.csv
%P 1
%C
%H
%Q
\end{verbatim}
}

この入力データで行おうとしていることは,以下の通りである.なお,サーバ側のプログラムに関しては,クライアントの起動と同タイミングで起動したものと仮定する.

\begin{enumerate}
      \item Cコマンドを実行し,名簿の件数を表示(0件)
      \item 1行分の名簿データを与え,サーバへ保存
      \item 再びCコマンドを実行し,名簿の件数が正しく増分していることを確認(1件)
      \item Rコマンドを実行し,大量の名簿データをまとめて追加
      \item Pコマンド(引数1)を実行し,先頭の名簿データをサーバから取得して表示
      \item Cコマンドを実行し,ファイルから読み込んだデータがサーバ側に保存されていることを確認(2887件)
      \item Hコマンドを実行し,これまでの操作履歴をサーバ側から取得して表示
      \item Qコマンドを実行し,クライアント側のプログラムを終了
\end{enumerate}

クライアント側では以下のような出力を得る.

\begin{verbatim}
0 profile(s)
1 profile(s)
success: cmd_read
Id    : 5100046
Name  : The Bridge
Birth : 1845-11-02
Addr  : 14 Seafield Road Longman Inverness
Com.  : SEN Unit 2.0 Open

2887 profile(s)
1: %C
2: %C
3: %P 1
4: %C
5: %H
\end{verbatim}

この出力結果から,意図した操作が行われていることがわかる.

\subsection{簡易認証プログラムの使用法}

今回作成した簡易認証プログラムでは,名簿管理プログラム同様一行ずつのコマンド入力を受け付けるようになっている.

具体例として,クライアント側に以下に示す入力を与えた場合について説明する.

\begin{verbatim}
  Status
  Edit keita
  Login keita password
  Logout
  Register keita 1111 1112
  Register keita 1111 1111
  Status
  Edit yamada
  Login yamada 1111
  Logout
  Login yamada 1112
  Login yamada 1111
  Status
\end{verbatim}

前述した入力を行うと,以下のような出力が得られる.

\begin{verbatim}
  Status
  Status: Guest
  Edit keita
  ServerError Invalid_Token
  Login keita password
  ServerError User_Not_Found
  Logout
  ClientError Not_Authenticated
  Register keita 1111 1112
  ServerError Password_Unmatched
  Register keita 1111 1111
  Register success!
  Status
  Status: Authenticated
  Edit yamada
  Edit success!
  Login yamada 1111
  ClientError Already_Authenticated
  Logout
  Logout success!
  Login yamada 1112
  ServerError Password_Is_Incorrected
  Login yamada 1111
  Login success!
  Status
  Status: Authenticated
\end{verbatim}

この一連の入力で何が行われているのかを順に説明する.

\begin{enumerate}
      \item Status コマンドで現在の状態を確認する.(ログインしていないため,ゲスト状態であると出力される)
      \item Edit コマンドで名前を編集しようとする.(が,未ログインのため失敗)
      \item Login コマンドを実行し,ログインしようとする.(が,該当のユーザがサーバに存在しないため失敗)
      \item Logout コマンドを実行し,ログアウトしようとする.(が,ログイン中でないため失敗)
      \item Register コマンドを実行し,会員登録を行おうとする.(が,確認用パスワードが一致しないため失敗)
      \item Register コマンドを実行し,会員登録を行う.(今度は成功し,ログイン状態に切り替わる)
      \item Status コマンドで現在の状態を確認する.(先ほど会員登録を行なったため,ログイン状態であると出力される)
      \item Edit コマンドで名前を編集する.(今度は成功)
      \item Login コマンドを実行し,ログインしようとする.(が,既にログイン済みのため失敗)
      \item Logout コマンドを実行し,ログアウトする(成功)
      \item Login コマンドを実行し,ログインしようとする.(が,パスワードが異なるため失敗)
      \item Login コマンドを実行し,ログインする.(成功)
      \item Status コマンドで現在の状態を確認する.(先ほどログインを行なったため,ログイン状態であると出力される)
\end{enumerate}


\section{プログラムの作成過程における考察}

\subsection{名簿管理プログラム}

ここでは,名簿管理プログラムをサーバクライアントモデルで実装する際に検討した内容,及び考察した内容について述べる.

\subsubsection{新規実装コマンドについての考察}

今回の名簿管理プログラムでは,自分独自のコマンドとして,サーバ側へのコマンドの入力履歴を取得するためのHコマンドを実装した.この機能の実装において工夫した点は,入力履歴データの管理方法である.当初はサイズの大きな配列を用意して,ひたすら末尾にデータを追加していくという手法を採用していたが,Hコマンドは最新の数件のみを返すという仕様であり,不要なデータを保持しているという問題点があった.このため,配列のサイズの上限を小さめの値にし,古いデータから順に削除するキュー構造として取り扱うようにし,メモリ空間の節約を図った.

\subsubsection{容量の大きいデータの通信についての考察}

WコマンドやPコマンドについて,当初の実装では{\tt send}関数で送れるバッファサイズの上限が存在するために,大量のテキストをクライアント側にまとめて送信することが困難であり,Cコマンドで件数を取得したあと,件数分のリクエストを送り,一件ずつ名簿データを取得しクライアント側で結合するという手法を取っていた.つまり,サーバ側はコマンドとともに名簿データの番号を受け取り,該当のデータを一件を返すという実装になっていた.しかし,この実装ではクライアント側の責務が大きくなったり(Pコマンドの取得範囲制御もクライアント側で行う必要がある),通信回数がデータの件数と同じになってしまい通信効率が悪い,という問題点があった.そこで,サーバ側のプログラムにおける{\tt send}関数の部分やクライアント側のプログラムにおける{\tt recv}関数の部分を改良し,通信するデータサイズが大きくなったとしても,バッファサイズで分割して必要最低限の回数で通信できるようにした.また,これらの処理は通信部に隠蔽しているため,コマンド処理側から通信するデータサイズを意識する必要がなくなり,WコマンドやPコマンドの処理をサーバ側に寄せることが可能となった.

\subsubsection{モジュール分割についての考察}

このプログラム(クライアント部)では,ソースコードを複数のモジュールに分割し(通信部,コマンド実行部,その他),
{\tt Makefile}を活用した分割コンパイルを行うようにした.
こうすることで,ソースコードの見通しがよくなったり,
通信部モジュールのみを簡易認証プログラムの部品として再利用できるようになったりと,様々な恩恵を得られることができた.

\subsection{簡易認証プログラム}

ここでは,名簿管理プログラムをサーバクライアントモデルで実装する際に検討した内容,及び考察した内容について述べる.

\subsubsection{責務の分割についての考察}

このプログラムでは,サーバとクライアント間での責務の分割を意識し,クライアント側は入力の受付,サーバ側は入力内容の検証(バリデーション)を行うようにした.
クライアント側でも入力のバリデーションはできるのではないか(クライアントとサーバに分ける意味はあるのか)と思う人もいるかもしれないが,
現実的にはクライアント側のプログラムはブラウザの検証ツールなどを用いればユーザ側で好きなように改造できるため,必ずサーバ側で入力の検証を行う必要があると考える.

\section{得られた結果に関する考察}

\subsection{名簿管理プログラム}

\subsubsection{不足機能に関する考察}

今回のプログラムでは,サーバ側にRコマンド用のインタフェースが備わっておらず,
サーバ側で一行の名簿データの入力を受け取るという仕様となっているため,
クライアント側のファイルに存在する名簿データの件数と同じ回数の通信を行わなければならず,
コマンドの入力履歴にも記録されないという問題点がある.
この問題を解決するには{\tt http}プロトコルにおけるリクエストボディのようなものをプロトコルとして規定し,
Rコマンドとそれに付随するクライアント側のファイルの内容をコマンド部とデータ部を同時に送信できるようにすべきだと考える.
(つまり,サーバ側のリクエスト解析部の大幅な改修が必要となる)

\subsubsection{エラー処理についての考察}

今回のプログラムの通信部では,一つ一つのシステムコール関数の結果を変数として格納しておき,
その結果がエラーになった場合はエラー内容を出力するようにしている.
一つ一つの手順ごとに{\tt if}文で条件分岐がなされており,多少冗長に感じるかもしれないが,
このような実装にすることによって,エラーが起きた際の原因の切り分けが行いやすいというメリットがある.
なお,エラー出力の際には{\tt printf}関数ではなく{\tt perror}関数を用いるようにし,
プロセスで発生した最新のエラーの原因である{\tt errno}の値を出力できるようにした.

\subsection{簡易認証プログラム}

\subsubsection{不足機能に関する考察}
このプログラムでは,クライアントで入力されたパスワードを平文のままサーバ上に保存しており,
サーバ側からユーザの認証情報が筒抜けとなっているという問題点がある.
これを解決するためには,クライアント側にハッシュ関数を実装し,サーバ側に生のパスワードの文字列が送られないようにする必要がある.

また,トークンの扱いについても,現実世界で使用されているものよりずっと簡略化したものなので,
例えば,ステータスを確認する際にクライアント側のトークン存在の有無だけで認証済みかどうかを判断しているが,
実際はサーバ側にトークン情報を送り,そのトークンが「有効」かどうかを判別する機能も必要であると考える.
(実際問題,現実世界の認証に使われるトークンには有効期限が定められているものが多い)

\subsubsection{エラー処理についての考察}

このプログラムでは,あるエラーがクライアント側で起きたのかサーバ側で起きたのかの切り分けを行いやすくするため,
全てのエラーメッセージに接頭辞をつけるようにした.({\tt ClientError, ServerError}) 
これは,{\tt http}プロトコルで通信を行なった際のエラーメッセージがクライアント側の問題かサーバ側の問題なのかを番号ですぐに区別できる
(400番台はクライアント側のエラー,500番台はサーバ側のエラー)という点を参考にした.

\section{作成したプログラム}

\subsection{名簿管理プログラムの通信部(クライアント)} \label{sec:client.c}

\begin{verbatim}
      #include "client.h"

      int request(char *request, char *response) {
        struct hostent *hp;
        hp = gethostbyname("localhost");
      
        if (hp == NULL) {
          printf("host not found\n");
          return -1;
        }
      
        // create socket
        int soc = socket(AF_INET, SOCK_STREAM, 0);
        if (soc == -1) {
          printf("failed to create socket\n");
          return -1;
        }
      
        struct sockaddr_in sa;
        sa.sin_family = hp->h_addrtype;
        sa.sin_port = htons(PORT_NO);
        bzero((char *)&sa.sin_addr, sizeof(sa.sin_addr));
        memcpy((char *)&sa.sin_addr, (char *)hp->h_addr, hp->h_length);
      
        int connect_result = connect(soc, 
                  (struct sockaddr *)&sa, sizeof(sa));
        if (connect_result == -1) {
          printf("failed to connect\n");
          return -1;
        }
      
        // send request
        int send_result = 0;
        int sended_bytes = 0;
        do {
          send_result = send(soc, request + sended_bytes,
                             fmin(BUF_SIZE, 
                             strlen(request + sended_bytes) + 1), 0);
          sended_bytes += send_result;
          if (send_result == -1) {
            perror("failed to send");
            break;
          }
        } while (send_result == BUF_SIZE);
      
        // receive response
        int recv_result = 0;
        sprintf(response, "");
        do {
          char tmp[BUF_SIZE] = "";
          recv_result = recv(soc, tmp, BUF_SIZE, 0);
          strcat(response, tmp);
          if (recv_result == -1) {
            printf("failed to receive\n");
            return -1;
          }
        } while (recv_result == BUF_SIZE);
      
        close(soc);
      
        return 0;
      }
\end{verbatim}

\subsection{名簿管理プログラムの標準入力解析部(クライアント)} \label{sec:meibo_client.c}

\begin{verbatim}
#include "meibo_client.h"

int main() {
  char line[INPUT_MAX + 1];
  while (get_line_fp(stdin, line)) {
    parse_line(line);
  }
  return 1;
}

void parse_line(char *line) {
  if (*line == '%') {
    char *exec[] = {"", "", "", "", ""};
    split(line + 1, exec, ' ', 5);
    exec_command_str(exec);
  } else {
    char response[BUF_SIZE];
    request(line, response);
  }
}

void exec_command_str(char *exec[]) {
  if (!strcmp("Q", exec[0])) {
    cmd_quit();
  } else if (!strcmp("C", exec[0])) {
    cmd_check();
  } else if (!strcmp("P", exec[0])) {
    cmd_print(exec[1]);
  } else if (!strcmp("R", exec[0])) {
    cmd_read(exec[1]);
  } else if (!strcmp("W", exec[0])) {
    cmd_write(exec[1]);
  } else if (!strcmp("H", exec[0])) {
    cmd_history();
  } else {
    fprintf(stderr, "Invalid command %s: ignored.\n", exec[0]);
  }
  return;
}

\end{verbatim}

\subsection{名簿管理プログラムのコマンド処理部(クライアント)} \label{sec:commands.c}

\begin{verbatim}
#include "commands.h"

void cmd_quit() { exit(0); }

void cmd_check() {
  char response[BUF_SIZE];
  request("%C", response);
  printf("%s profile(s)\n", response);
}

void cmd_read(char *path) {
  FILE *fp;
  fp = fopen(path, "r");

  if (fp == NULL) {
    fprintf(stderr, "Could not open file: %s\n", path);
    return;
  }

  char line[INPUT_MAX + 1];
  while (get_line_fp(fp, line)) { /* fp を引数に追加 */
    char response[BUF_SIZE];
    request(line, response);
  }

  fclose(fp);

  printf("success: cmd_read\n");

  return;
};

void cmd_write(char *path) {
  FILE *fp;
  if (*path == '\0') {
    path = "data/output.csv";
  }

  fp = fopen(path, "w");
  if (fp == NULL) {
    fprintf(stderr, "Could not open file: %s\n", path);
    return;
  }
    
  char response[RESPONSE_BUF_SIZE];
  request("%W", response);

  fprintf(fp, response);
  fclose(fp);

  printf("write success: %s\n", path);

  return;
}

void cmd_print(char *num) {
  char query[BUF_SIZE];
  sprintf(query,"%%P %s", num);
  char response[RESPONSE_BUF_SIZE];
  request(query, response);
  printf(response);
  return;
}

void cmd_history() {
  char query[BUF_SIZE] = "%H";
  char response[BUF_SIZE];
  
  request("%H", response);
  printf("%s\n", response);
}
\end{verbatim}

\subsection{名簿管理プログラムの通信部(サーバ)} \label{sec:server.c}

\begin{verbatim}
      int soc = socket(AF_INET, SOCK_STREAM, 0);
      if (soc == -1) {
        printf("failed to create socket\n");
        return 1;
      }
    
      // bind port
      struct sockaddr_in s_sa;
      memset((char *)&s_sa, 0, sizeof(s_sa));
      s_sa.sin_family = AF_INET;
      s_sa.sin_addr.s_addr = htonl(INADDR_ANY);
      s_sa.sin_port = htons(PORT_NO);
      int bind_result = 
          bind(soc, (struct sockaddr *)&s_sa, sizeof(s_sa));
      if (bind_result == -1) {
        perror("failed to bind port\n");
        return 1;
      }
    
      // listen socket
      int listen_result = listen(soc, 5);
      if (listen_result == -1) {
        perror("failed to listen");
        return 1;
      }
    
      while (1) {
        // accept request
        struct sockaddr_in c_sa;
        int c_sa_len = sizeof(c_sa);
        int fd = accept(soc, (struct sockaddr *)&c_sa, &c_sa_len);
        if (fd == -1) {
          perror("failed to accept");
          return 1;
        }
    
        // receive response
        char request[BUF_SIZE] = "";
        int recv_result = 0;
        do {
          char tmp[BUF_SIZE] = "";
          recv_result = recv(fd, tmp, BUF_SIZE, 0);
          strcat(request, tmp);
          if (recv_result == -1) {
            printf("failed to receive\n");
            return -1;
          }
        } while (recv_result == BUF_SIZE);
    
        char response[RESPONSE_BUF_SIZE] = "";
        parse_line(request, response);
    
        int send_result = 0;
        int sended_bytes = 0;
        do {
          send_result =
              send(fd, response + sended_bytes,
                   fmin(BUF_SIZE, 
                      strlen(response + sended_bytes) + 1), 0);
          sended_bytes += send_result;
          if (send_result == -1) {
            perror("failed to send");
            break;
          }
        } while (send_result == BUF_SIZE);
    
        close(fd);
      }
      close(soc);
\end{verbatim}

\subsection{名簿管理プログラムのリクエスト解析部(サーバ)} \label{sec:parse_request.c}

\begin{verbatim}
      void parse_line(char *line, char *response) {
            if (*line == '%') {
              if (strlen(history[HISTORY_MAX - 1]) != 0) {
                for (int i = 0; i < HISTORY_MAX - 1; i++) {
                  strcpy(history[i], history[i + 1]);
                }
                strcpy(history[HISTORY_MAX - 1], line);
              } else {
                for (int i = 0; i < HISTORY_MAX; i++) {
                  if (strlen(history[i]) == 0) {
                    strcpy(history[i], line);
                    break;
                  }
                }
              }
          
              char *exec[] = {"", "", "", "", ""};
              split(line + 1, exec, ' ', 5);
              exec_command_str(exec, response);
            } else {
              if (profile_data_nitems >= MAX_PROFILES) {
                sprintf(response, "The upper limit has been reached");
              } else {
                new_profile(
                  &profile_data_store[profile_data_nitems], line);
                sprintf(response, "new profile created");
              }
            }
          }
          
          void new_profile(struct profile *p, char *line) {
            char *error;
          
            p->id = 0;
            strncpy(p->school_name, "", 70);
            p->create_at.y = 0;
            p->create_at.m = 0;
            p->create_at.d = 0;
            strncpy(p->place, "", 70);
          
            char *ret[5];
            if (split(line, ret, ',', 5) < 5) {
              printf("要素が不足しています\n");
              return;  //不都合な入力の際は処理を中断する
            }
          
            p->id = strtoi(ret[0], &error);
            if (*error != '\0') {
              printf("idの入力に失敗しました\n");
              return;  //数字が入力されていない場合は処理を中断する
            }
          
            strncpy(p->school_name, ret[1], LIMIT);
          
            char *date[3];
            int d[3] = {};
            if (split(ret[2], date, '-', 3) < 3) {
              printf("設立日の入力に失敗しました\n");
              return;  //不都合な入力の際は処理を中断する
            }
            int i;
            for (i = 0; i < 3; i++) {
              d[i] = strtoi(date[i], &error);
              if (*error != '\0') {
                printf("設立日の入力に失敗しました\n");
                return;  //数字が入力されていない場合は処理を中断する
              }
            }
            p->create_at.y = d[0];
            p->create_at.m = d[1];
            p->create_at.d = d[2];
          
            strncpy(p->place, ret[3], LIMIT);
          
            p->note = (char *)malloc(strlen(ret[4]) + 1);  //動的にメモリを確保
            strcpy(p->note, ret[4]);
          
            profile_data_nitems++;
          }
          
          void exec_command_str(char *exec[], char *response) {
            char *error;
          
            if (!strcmp("C", exec[0])) {
              cmd_check(response);
            } else if (!strcmp("P", exec[0])) {
              int param_num = strtoi(exec[1], &error);
              if (*error != '\0') {
                printf("パラメータは整数にしてください\n");
                return;
              }
              cmd_print(param_num, response);
            } else if (!strcmp("W", exec[0])) {
              cmd_write(response);
            } else if (!strcmp("H", exec[0])) {
              cmd_history(response);
            } else {
              fprintf(stderr, "Invalid command %s: ignored.\n", exec[0]);
            }
            return;
          }   
\end{verbatim}

\subsection{名簿管理プログラムのコマンド処理部(サーバ)} \label{sec:commands_server.c}

\begin{verbatim}
      void cmd_check(char *response) { 
        sprintf(response, "%d", profile_data_nitems);
      }

      void cmd_print(int p, char *response) {
        if (p > 0) {
          if (p > profile_data_nitems)
            p = profile_data_nitems;  //登録数よりも多い場合,要素数に合わせる
      
          int i;
          for (i = 0; i < p; i++) {
            print_profile(i, response);
          }
        } else if (p == 0) {
          int i;
          for (i = 0; i < profile_data_nitems; i++) {
            print_profile(i, response);
          }
        } else {
          if (abs(p) > profile_data_nitems)
            p = profile_data_nitems;  //登録数よりも多い場合,要素数に合わせる
      
          int i;
          for (i = profile_data_nitems - abs(p);
           i <= profile_data_nitems - 1; i++) {
            print_profile(i, response);
          }
        }
      }
      
      void print_profile(int index, char *response) {
        struct profile *p = profile_data_store_ptr[index];
        char tmp[BUF_SIZE] = "";
        sprintf(tmp, "Id    : %d\n", p->id);
        strcat(response, tmp);
        sprintf(tmp, "Name  : %s\n", p->school_name);
        strcat(response, tmp);
        sprintf(tmp, "Birth : %04d-%02d-%02d\n", 
          p->create_at.y, 
          p->create_at.m,
          p->create_at.d);
        strcat(response, tmp);
        sprintf(tmp, "Addr  : %s\n", p->place);
        strcat(response, tmp);
        sprintf(tmp, "Com.  : %s\n\n", p->note);
        strcat(response, tmp);
      }
      
      void cmd_write(char *response) {
        for (int i = 0; i < profile_data_nitems; i++) {
          struct profile *p = profile_data_store_ptr[i];
          char tmp[BUF_SIZE] = "";
          sprintf(tmp, 
            "%d,%s,%04d-%d-%d,%s,%s\n", 
            p->id, 
            p->school_name,
            p->create_at.y, 
            p->create_at.m, 
            p->create_at.d, 
            p->place, p->note);
          strcat(response, tmp);
        }
      }
      
      void cmd_history(char *response) {
        for (int i = 0; i < HISTORY_MAX; i++) {
          if (strlen(history[i]) != 0) {
            char tmp[BUF_SIZE];
            sprintf(tmp, "%d: %s\n", i + 1, history[i]);
            strcat(response, tmp);
          }
        }
      }
\end{verbatim}

\subsection{簡易認証プログラムのコマンド処理部(クライアント)} \label{sec:auth_commands_client.c}

\begin{verbatim}
  void cmd_register(char *name, char *password, char *password_confirm) {
    char req[BUF_SIZE] = "";
    char res[BUF_SIZE] = "";
  
    sprintf(req, "Register %s %s %s", name, password, password_confirm);
    request(req, res);
  
    char *result[] = {"", ""};
    split(res, result, ' ', 5);
  
    if (!strcmp(result[0], "ServerError")) {
      fprintf(stderr, "%s %s", result[0], result[1]);
    } else if (!strcmp(result[0], "Success")) {
      printf("Register success!\n");
      strncpy(token, result[1], LIMIT);
    }
  }
  
  void cmd_status() {
    if (!strcmp(token, "")) {
      printf("Status: Guest\n");
    } else {
      printf("Status: Authenticated\n");
    }
  }
  
  void cmd_login(char *name, char *password) {
    if(strcmp(token, "") != 0){
      printf("ClientError Already_Authenticated\n");
      return;
    }
  
    char req[BUF_SIZE] = "";
    char res[BUF_SIZE] = "";
  
    sprintf(req, "Login %s %s", name, password);
    request(req, res);
  
    char *result[] = {"", ""};
    split(res, result, ' ', 5);
  
    if (!strcmp(result[0], "ServerError")) {
      fprintf(stderr, "%s %s", result[0], result[1]);
    } else if (!strcmp(result[0], "Success")) {
      printf("Login success!\n");
      strncpy(token, result[1], LIMIT);
    }
  }
  
  void cmd_edit(char *new_name) {
    char req[BUF_SIZE] = "";
    char res[BUF_SIZE] = "";
  
    sprintf(req, "Edit %s %s", new_name, token);
    request(req, res);
  
    char *result[] = {"", ""};
    split(res, result, ' ', 5);
  
    if (!strcmp(result[0], "ServerError")) {
      fprintf(stderr, "%s %s", result[0], result[1]);
    } else if (!strcmp(result[0], "Success")) {
      printf("Edit success!\n");
    }
  }
  
  void cmd_logout() {
    if(!strcmp(token, "")){
      printf("ClientError Not_Authenticated\n");
      return;
    }
  
    sprintf(token, "");
    printf("Logout success!\n");
  }
\end{verbatim}

\subsection{簡易認証プログラムのコマンド実現部(サーバ側)} \label{sec:auth_commands_server.c}

\begin{verbatim}
  
void cmd_register(char *name, char *password, char *password_confirm,
                  char *response) {
  if (find_user_by_name(name) != -1) {
    sprintf(response, "ServerError Already_Registered\n");
    return;
  }
  if (strcmp(password, password_confirm) != 0) {
    sprintf(response, "ServerError Password_Unmatched\n");
    return;
  }
  char token[LIMIT] = "";
  create_user(name, password, token);
  sprintf(response, "Success %s", token);
  return;
}

void create_user(char *name, char *password, char *token) {
  struct user *user = &user_data_store[user_count];
  strncpy(user->name, name, LIMIT);
  strncpy(user->password, password, LIMIT);
  generate_token(token);
  strncpy(user->token, token, LIMIT);
  user_count += 1;
  return;
}

void generate_token(char *token) {
  char candidates[] =
      "ABCDEFGHIJKLMNOPQRSTUVWXYZabcdefghijklmnopqrstuvwxyz0123456789";
  srand((unsigned)time(NULL));
  for (int i = 0; i < LIMIT; i++) {
    int index = random() % (sizeof(candidates) - 1);
    token[i] = candidates[index];
  }
  token[LIMIT - 1] = '\0';
  return;
}

void cmd_login(char *name, char *password, char *response) {
  int user_index = find_user_by_name(name);

  if (user_index == -1) {
    sprintf(response, "ServerError User_Not_Found\n");
    return;
  }
  if (strcmp(password, (&user_data_store[user_index])->password) != 0) {
    sprintf(response, "ServerError Password_Is_Incorrected\n");
    return;
  }

  sprintf(response, "Success %s", (&user_data_store[user_index])->token);
  return;
}

void cmd_edit(char *new_name, char *token, char *response) {
  int user_index = find_user_by_token(token);
  if (user_index == -1) {
    sprintf(response, "ServerError Invalid_Token\n");
    return;
  }

  strncpy((&user_data_store[user_index])->name, new_name, LIMIT);
  sprintf(response, "Success %s\n", new_name);
}

int find_user_by_name(char *name) {
  for (int i = 0; i < user_count; i++) {
    if (!strcmp(name, (&user_data_store[i])->name)) {
      return i;
    }
  }
  return -1;
}

int find_user_by_token(char *token) {
  for (int i = 0; i < user_count; i++) {
    if (!strcmp(token, (&user_data_store[i])->token)) {
      return i;
    }
  }
  return -1;
}
\end{verbatim}

\end{document}